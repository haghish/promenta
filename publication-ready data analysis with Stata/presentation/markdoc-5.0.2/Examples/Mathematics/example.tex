
\documentclass{article}
\usepackage{geometry} 
\usepackage{booktabs}         %for tables
%\geometry{letterpaper} 
\usepackage{graphicx}
\usepackage{amssymb}
\usepackage{hyperref}         %use hyperlink
\usepackage{epstopdf}
\DeclareGraphicsRule{.tif}{png}{.png}{`convert #1 `dirname #1`/`basename #1 .tif`.png}
\makeatletter
\def\verbatim@font{\ttfamily\scriptsize}
\makeatother
\begin{document}
\title{Testing Mathematical Notations in MarkDoc}
\author{E. F. Haghish } 
\maketitle


Use a single ``\$'' sign for writing inline mathematical notations. For
example, \(f(x)=\sum_{n=0}^\infty\frac{f^{(n)}(a)}{n!}(x-a)^n\) would be
rendered inline with the text paragraph. Use double dollar signs
``\$\$'' for placing the notations on a separate lines:

\[Y_i = \beta_0 + \beta_1 X_i + \epsilon_i\]

\begin{enumerate}
\def\labelenumi{\arabic{enumi}.}
\item
  Since the notations appear in comments, they will not be interpreted
  by Stata as global macros.
\item
  Place a backslash before the ``\$'' if you are using them in the
  document, but not for rendering mathematical notations. The backslash
  will not appear in the dynamic document.
\item
  You can also write dynamic mathematical notations using the
  \textbf{\texttt{txt}} command.

\begin{verbatim}
  .  local a = 10
\end{verbatim}
\end{enumerate}

\[ \beta_1 = 10 \]

Note that when you write inline mathematical notations, there should be
\textbf{NO SPACE} between the dollar sign and the notation. However, if
you are placing your notations on a separate line, there should be no
problem.

\end{document}



